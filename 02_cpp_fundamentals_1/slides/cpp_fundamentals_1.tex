\documentclass[12pt,aspectratio=169]{beamer}

\usetheme{metropolis}

\definecolor{mDarkBrown}{HTML}{FF5722}
\definecolor{mDarkTeal}{HTML}{263238}
\definecolor{mLightBrown}{HTML}{FF5722}

\usepackage{booktabs}
\usepackage{graphicx}
\usepackage{hyphenat}

\usepackage{minted}
\usemintedstyle{tango}
\newminted[bash]{bash}{%
    autogobble,
    bgcolor=mDarkTeal!10,
    linenos
}
\newminted[cpp]{cpp}{%
    autogobble,
    bgcolor=mDarkTeal!10,
    linenos
}

\usepackage{polyglossia}
\setdefaultlanguage[variant=british]{english}
\usepackage[english=british]{csquotes}

\defaultfontfeatures{Ligatures=TeX}
\setmainfont{Lucida Sans OT}
\setsansfont[Scale=MatchLowercase]{Lucida Sans OT}
\setmonofont[Scale=MatchLowercase]{Lucida Console DK}

\usepackage{mathspec}
\setmathsfont(Digits,Latin,Greek)[Numbers={Lining,Proportional}]{Lucida Bright Math OT}

\author{Gianluca Campanella}
\date{}



\title{C++ fundamentals}
\subtitle{Part 1}

\begin{document}

\maketitle

\section{Variables and types}

\begin{frame}[fragile]{Variables}
    \vspace{1em}
    \begin{itemize}
        \item Used to store mutable values
        \item Must have a \alert{type} and a \alert{name}
    \end{itemize}
    \begin{cpp}
        int x = 0;
    \end{cpp}
    \begin{center}
        \begin{tabular}{lll}
            \toprule
            \textbf{Type}            & \textbf{Use}                & \textbf{Values} \\
            \midrule
            \mintinline{cpp}{bool}   & Booleans                    & \mintinline{cpp}{true}, \mintinline{cpp}{false} \\
            \mintinline{cpp}{char}   & Characters                  & \mintinline{cpp}{'a'}, \mintinline{cpp}{'b'} \\
            \mintinline{cpp}{int}    & Integers                    & \mintinline{cpp}{0}, \mintinline{cpp}{-1} \\
            \mintinline{cpp}{double} & Floating\hyp{}point numbers & \mintinline{cpp}{0.1}, \mintinline{cpp}{1e-3} \\
            \bottomrule
        \end{tabular}
    \end{center}
\end{frame}

\begin{frame}[fragile]{Arithmetic operators}
    \vspace{1em}
    \begin{center}
        \begin{tabular}{ll}
            \toprule
            \textbf{Syntax}         & \textbf{Operation} \\
            \midrule
            \mintinline{cpp}{a = b} & Assignment \\
            \mintinline{cpp}{a + b} & Addition \\
            \mintinline{cpp}{a - b} & Subtraction \\
            \mintinline{cpp}{a * b} & Multiplication \\
            \mintinline{cpp}{a / b} & Division \\
            \mintinline{cpp}{a % b} & Modulo \\
            \mintinline{cpp}{++a}   & Prefix increment \\
            \mintinline{cpp}{a++}   & Postfix increment \\
            \mintinline{cpp}{--a}   & Prefix decrement \\
            \mintinline{cpp}{a--}   & Postfix decrement \\
            \bottomrule
        \end{tabular}
    \end{center}
\end{frame}

\begin{frame}[fragile]{Integers}
    \begin{block}{\mintinline{cpp}{int}s store integers}
        \begin{cpp}
            int x = 2.5;
        \end{cpp}
    \end{block}
    \begin{block}{Watch for integer division!}
        \begin{cpp}
            int x = 2 / 3;
            double y = 2.0 / 3.0;
        \end{cpp}
    \end{block}
\end{frame}

\begin{frame}[fragile]{Comparison operators}
    \vspace{1em}
    \begin{center}
        \begin{tabular}{ll}
            \toprule
            \textbf{Syntax}          & \textbf{Operation} \\
            \midrule
            \mintinline{cpp}{a == b} & Equal to \\
            \mintinline{cpp}{a != b} & Not equal to \\
            \mintinline{cpp}{a < b}  & Less than \\
            \mintinline{cpp}{a > b}  & Greater than \\
            \mintinline{cpp}{a <= b} & Less than or equal to \\
            \mintinline{cpp}{a >= b} & Greater than or equal to \\
            \bottomrule
        \end{tabular}
    \end{center}
\end{frame}

\begin{frame}[fragile]{Assignment and equality operators}
    \begin{block}{Assignment operator}
        \begin{cpp}
            int x = 0;
            int y = x - 1;
        \end{cpp}
    \end{block}
    \begin{block}{Equality operator}
        \begin{cpp}
            int x = 0;
            int y = -1;
            bool z = y == x - 1;
        \end{cpp}
    \end{block}
\end{frame}

\begin{frame}[fragile]{Logical operators}
    \vspace{1em}
    \begin{center}
        \begin{tabular}{ll}
            \toprule
            \textbf{Syntax}           & \textbf{Operation} \\
            \midrule
            \mintinline{cpp}{!a}      & $\neg a$ \\
            \mintinline{cpp}{a && b}  & $a \land b$ \\
            \mintinline{cpp}{a || b}  & $a \lor b$ \\
            \midrule
            \mintinline{cpp}{a != b}  & $a \oplus b$ \\
            \mintinline{cpp}{!a || b} & $a \implies b$ \\
            \bottomrule
        \end{tabular}
    \end{center}
\end{frame}

\begin{frame}[fragile]{Spot the difference!}
    \begin{center}
        {\Large%
         What's the difference between \\[1em]
         \mintinline{cpp}{a!=b} and \mintinline{cpp}{a=!b}?}
    \end{center}
\end{frame}

\begin{frame}[fragile]{Type conversion and casting}
    \begin{block}{Type conversion (implicit)}
        \begin{cpp}
            double d = 3.5;
            int i = d; // Truncated to 3 without warning!
            cout << d << " " << i << endl;
        \end{cpp}
    \end{block}
    \begin{block}{Casting (explicit)}
        \begin{cpp}
            double d = 3.5;
            cout << d << " " << (int) d << endl;
        \end{cpp}
    \end{block}
\end{frame}

\section{Flow control}

\begin{frame}[fragile]{If statements}
    \begin{cpp}
        if (condition1)
        {
            // ...
        }
        else if (condition2)
        {
            // ...
        }
        else
        {
            // ...
        }
    \end{cpp}
\end{frame}


\begin{frame}[fragile]{While loops}
    \begin{cpp}
        while (condition)
        {
            // ...
        }
    \end{cpp}
    \vspace{-1em}
    \begin{itemize}
        \item \mintinline{cpp}{while} iterates until \mintinline{cpp}{condition}
              is \mintinline{cpp}{false}
        \item Careful with infinite loops!
    \end{itemize}
\end{frame}

\begin{frame}[fragile]{For loops}
    \vspace{0.5em}
    \begin{cpp}
        for (initialisation; condition; update)
        {
            // ...
        }
    \end{cpp}
    \vspace{-2.5em}
    \begin{cpp}
        initialisation; // Executed only once
        while (condition)
        {
            // ...
            update;
        }
    \end{cpp}
\end{frame}

\section{Functions}

\begin{frame}[fragile]{Functions}
    \begin{onlyenv}<1>
        \begin{itemize}
            \item Functions encapsulate (reusable) bits of computation
            \item They're defined in terms of their arguments and return value
        \end{itemize}
        \begin{cpp}
            return_type function(type1 arg1, /*, ... */)
            {
                // ...
                return value_of_return_type;
            }
        \end{cpp}
    \end{onlyenv}
    \begin{onlyenv}<2>
        \begin{block}{Example}
            \begin{cpp}
                int power(int x, int n); // Prototype

                int power(int x, int n)  // Declaration
                {
                    int result = 1;
                    for (int i = 1; i <= n; i++)
                    {
                        result *= x;
                    }
                    return result;
                }
            \end{cpp}
        \end{block}
    \end{onlyenv}
\end{frame}

\begin{frame}[fragile]{Argument passing}
    \begin{block}{Passing by value}
        \begin{itemize}
            \item Argument values are \alert{copied} by default
            \item[$\rightarrow$] Modifying them inside the function has no
                                 effect outside
        \end{itemize}
    \end{block}
    \begin{block}{Passing by reference}
        \begin{itemize}
            \item Arguments can be referenced using \mintinline{cpp}{int& x}
            \item[$\rightarrow$] Modifying them inside the function changes them
                                 outside too!
            \item Use \mintinline{cpp}{const int& x} to prevent changes
        \end{itemize}
    \end{block}
\end{frame}

\begin{frame}[fragile]{Polymorphism}
    \begin{cpp}
        int power(int x, int n);
        double power(double x, int n);
    \end{cpp}
    \vspace{-1em}
    \begin{itemize}
        \item These two functions can coexist!
        \item The compiler `knows' which one to call from the arguments:
              \begin{itemize}
                  \item \mintinline{cpp}{power(2, 5)} calls the first version
                  \item \mintinline{cpp}{power(2.5, 2)} calls the second version
              \end{itemize}
    \end{itemize}
\end{frame}

\end{document}

