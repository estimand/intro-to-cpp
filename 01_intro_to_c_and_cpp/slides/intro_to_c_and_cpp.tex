\documentclass[12pt,aspectratio=169]{beamer}

\usetheme{metropolis}

\definecolor{mDarkBrown}{HTML}{FF5722}
\definecolor{mDarkTeal}{HTML}{263238}
\definecolor{mLightBrown}{HTML}{FF5722}

\usepackage{booktabs}
\usepackage{graphicx}
\usepackage{hyphenat}

\usepackage{minted}
\usemintedstyle{tango}
\newminted[bash]{bash}{%
    autogobble,
    bgcolor=mDarkTeal!10,
    linenos
}
\newminted[cpp]{cpp}{%
    autogobble,
    bgcolor=mDarkTeal!10,
    linenos
}

\usepackage{polyglossia}
\setdefaultlanguage[variant=british]{english}
\usepackage[english=british]{csquotes}

\defaultfontfeatures{Ligatures=TeX}
\setmainfont{Lucida Sans OT}
\setsansfont[Scale=MatchLowercase]{Lucida Sans OT}
\setmonofont[Scale=MatchLowercase]{Lucida Console DK}

\usepackage{mathspec}
\setmathsfont(Digits,Latin,Greek)[Numbers={Lining,Proportional}]{Lucida Bright Math OT}

\author{Gianluca Campanella}
\date{}



\title{Introduction to C and C++}

\begin{document}

\maketitle

\begin{frame}{Thinking like a Computer Scientist}
    \only<1>{%
        \begin{block}{Computer Scientists\ldots}
            \begin{itemize}
                \item Use formal languages to denote ideas
                \item Design things, assembling components into systems
                \item Observe the behaviour of complex systems, form hypotheses,
                      and test predictions
            \end{itemize}
        \end{block}}
    \only<2>{%
        \begin{center}
            \Large%
            What's the most important skill \\
            for a Computer Scientist?
        \end{center}}
\end{frame}

\begin{frame}{Algorithms}
    \begin{itemize}
        \setlength{\itemsep}{0.75em}
        \item Step\hyp{}by\hyp{}step lists of instructions to solve a problem
        \item Can be represented in a specific notation (programs)
        \item Can be executed automatically by a computer
    \end{itemize}
\end{frame}

\begin{frame}{Programs}
    \begin{itemize}
        \setlength{\itemsep}{0.75em}
        \item Sequences of instructions that describes a computation
        \item Basic instructions include:
              \begin{itemize}
                  \item Input/output
                  \item Mathematical and logical operations
                  \item Conditional execution (if\hyp{}then)
                  \item Repetition
              \end{itemize}
    \end{itemize}
\end{frame}

\begin{frame}{Let's write an algorithm!}
    \begin{center}
        \Large%
        Compute the sum \\
        of all even numbers \\
        in a given list
    \end{center}
\end{frame}

\begin{frame}[fragile]{Our algorithm\ldots~in C++}
    \begin{cpp}
        const int n = 10;
        int x[n] = {0, 1, 1, 2, 3, 5, 8, 13, 21, 34};

        int result = 0;
        for (int i = 0; i < n; i++)
        {
            if (x[i] % 2 == 0)
            {
                result += x[i];
            }
        }
    \end{cpp}
\end{frame}

\begin{frame}{What is C++?}
    \begin{block}{C++ is\ldots}
        \begin{itemize}
            \item A general\hyp{}purpose programming language
            \item Imperative (like C) but also object\hyp{}oriented
            \item Designed for performance and flexibility
            \item Standardised by ISO
        \end{itemize}
    \end{block}
\end{frame}

\begin{frame}{Why C++?}
    \begin{itemize}
        \item Widely used
        \item Fast and portable
        \item Many libraries (especially for scientific computing)
        \item Full access to existing C and FORTRAN code
    \end{itemize}
\end{frame}

\begin{frame}{Compiling and running}
    \begin{block}{You'll need\ldots}
        \begin{enumerate}
            \item A \alert{text editor} (or \alert{IDE}) to write code in
            \item A \alert{compiler} to turn what you write into machine code
        \end{enumerate}
    \end{block}
\end{frame}

\begin{frame}[fragile]{Hello world!}
    \begin{cpp}
        #include <iostream>

        using namespace std;

        int main()
        {
            // Print "Hello world" to standard output
            cout << "Hello world!" << endl;
            return 0;
        }
    \end{cpp}
\end{frame}

\begin{frame}{Semicolons everywhere}
    \begin{block}{Don't forget the semicolon\alert{;}}
        \begin{itemize}
            \item Semicolons denote the end of a statement\alert{;}
            \item If you wanted, you could put all your code on a single line\alert{;}
            \item That's why the compiler needs semicolons\alert{;}
        \end{itemize}
    \end{block}
\end{frame}

\end{document}

